\documentclass[12pt]{article}

% Pretty much all of the ams maths packages
\usepackage{amsmath,amsthm,amssymb,amsfonts}

% Allows you to manipulate the page a bit
\usepackage[a4paper]{geometry}

% Norwegian letters
\usepackage[utf8]{inputenc}
\usepackage[norsk]{babel}

% Pulls the page out a bit - makes it look better (in my opinion)
\usepackage{a4wide}

% Removes paragraph indentation (not needed most of the time now)
\usepackage{parskip}

% Allows inclusion of graphics easily and configurably
\usepackage{graphicx}

% Provides ways to make nice looking tables
\usepackage{booktabs}

% Allows you to rotate tables and figures
\usepackage{rotating}

% Allows shading of table cells
\usepackage{colortbl}
% Define a simple command to use at the start of a table row to make it have a shaded background
\newcommand{\gray}{\rowcolor[gray]{.9}}

\usepackage{textcomp}

% Provides commands to make subfigures (figures with (a), (b) and (c))
\usepackage{subfigure}

% Typesets URLs sensibly - with tt font, clickable in PDFs, and not breaking across lines
\usepackage{url}

% Makes references hyperlinks in PDF output
%\usepackage{hyperref}

% Provides ways to include syntax-highlighted source code
\usepackage{listings}
\lstset{frame=single, basicstyle=\ttfamily}

% Provides Harvard-style referencing
\usepackage{natbib}
\bibpunct{(}{)}{;}{a}{,}{,}

% Provides good access to colours
\usepackage{color}
\usepackage{xcolor}

% Simple command I defined to allow me to mark TODO items in red
\newcommand{\todo}[1] {\textbf{\textcolor{red}{#1}}}

% Allows fancy stuff in the page header
\usepackage{fancyhdr}
\pagestyle{fancy}

% Vastly improves the standard formatting of captions
\usepackage[margin=10pt,font=small,labelfont=bf, labelsep=endash]{caption}

%Double spacing
%\usepackage{setspace}
%\doublespacing

% Standard title, author etc.
\title{STV1020: Kvalifiseringsprøve i R}
\author{}
\date{}
% Put text on the left-hand and right-hand side of the header
\fancyhead{}
\lhead{STV4020A}
\rhead{Kvalifiseringsprøve i R}
\chead{09.10.2017}
\begin{document}


\section*{Instruksjoner}
\begin{itemize}
\item Prøven skal besvares med et fungerende R-script som lastes opp i innleveringsmappen
p{\aa} Fronter.

\item Scriptet skal inneholde nødvendig kode for å besvare oppgavene samt kommentarer markert med \# som forklarer fremgangsmåten dere har valgt. Der oppgavene ber dere oppgi bestemte verdier eller tolkninger skal disse også oppgis som kommentarer i scriptet.

\item Husk at riktig kode er det viktigste; pass på å ikke bruk for lang tid på tolkninger.

\item Sørg for at koden er oversiktlig. For å skille oppgavene fra hverandre, anbefales overskrifter av typen: \\
\#\#\# Oppgave 1 \#\#\#\# i scriptet \\
\# Oppgave 1: \\ eller lignende.

\item Lykke til!
\end{itemize}

\vfill

\section*{Variabelforklaringer:}
\begin{description}
  \item[respondent{\_}id] Unik id for individuelle respondenter
  \item[steak{\_}prep] Hvordan respondenten foretrekker biff stekt
  \item[hhold\_income] Husholdningsinntekten til respondenten (i dollar)
  \item[age] Respondentens alder
  \item[smoke] Om respondenten r{\o}yker (1 = ja, 0 = nei)
  \item[alcohol] Om respondenten drikker alkohol (1 = ja, 0 = nei)
\end{description}

\newpage


\section*{Oppgaver}
\begin{enumerate}

\item Last inn data \textbf{steak{\_}survey.csv}. Enhetene i datasettet er respondenter i en survey. Du kan enten laste ned data fra fronter, linken: \\
\url{http://folk.uio.no/martigso/encrypt/} \\
eller direkte inn i R med: \\
\url{http://folk.uio.no/martigso/encrypt/steak_survey.csv}

\item Lag et stolpediagram av variabelen {\tt steak{\_}prep}. Kommenter hvilken verdi på variabelen som har høyest frekvens.

\item Lag en \textbf{ny variabel} -- {\tt steak{\_}prep2} -- som tar verdiene: \\
  \begin{itemize}
    \item ``Rare'' når {\tt steak{\_}prep} er ``Rare'' \textbf{eller} ``Medium rare''
    \item ``Medium'' når {\tt steak{\_}prep} er ``Medium''
    \item ``Well'' når {\tt
    steak{\_}prep} er ``Medium Well'' \textbf{eller} ``Well''
  \end{itemize}
Sjekk at variabelen ble kodet riktig med en tabell.

\item Gjør om den nye {\tt steak{\_}prep2} variabelen til en faktor, og sett
kategorien med flest enheter til referansekategori.

\item Vis hvordan du finner korrelasjonen mellom variblene {\tt smoke} og {\tt alcohol}. Oppgi korrelasjonen i en kommentar.

\item Lag et boxplot med {\tt steak{\_}prep2} på x-aksen og {\tt age} på y-aksen.
Hvilken kategori på {\tt steak{\_}prep2} har lavest median?

\item Estimer en multinomisk logistisk regresjon med {\tt steak{\_}prep2} som
avhengig variabel og {\tt age}, {\tt hhold{\_}income}, {\tt smoke} og {\tt
alcohol} som uavhengige variabler. Husk også å ta vare på informasjon om NA i
regresjonen. Kommenter kort hva retningen for begge koeffisientene til {\tt
smoke} betyr.

\item Vis hvordan du sjekker konfidensintervallene på 5\% nivå for effektene i
regresjonen fra oppgave 6. Er effekten av {\tt age} signifikant?

\item Legg inn predikerte \textbf{kategorier} (ikke sannsynligheter) i datasettet
fra regresjonen i oppgave 6. Lag en tabell over predikerte (forventede) og
faktiske verdier på {\tt steak{\_}prep2}. Kommenter kort hva tabellen viser.

\item Lag datasett (\emph{test set}) der alder går fra 18:90, {\tt
hhold{\_}income} er satt til median, {\tt smoke} er satt til 0 og {\tt alcohol}
er satt til 1. Legg så inn predikerte sannsynligheter (løs regresjonsligningen)
fra regresjonen (oppgave 7) i dette datasettet. Lag deretter et plot som har de
forventede sannsynlighetene til \emph{test settet} på y-aksen, alder på x-aksen
og fargede linjer for hver av kategoriene på {\tt steak{\_}prep2}.

\end{enumerate}

\end{document}
